% !TeX spellcheck = en_GB
\documentclass[aspectratio=169]{beamer}

%\usepackage[utf8]{inputenc}
%\usepackage[T1]{fontenc}
%\usepackage{lmodern}
\usepackage{fontspec}
\usepackage[english]{babel}
\usepackage{amsmath}
\usepackage{amsfonts}
\usepackage{amssymb}
\usepackage{graphicx}
\usepackage{bigints}
\usepackage{verbatim}
\usepackage{caption}
\usepackage{mathtools}
\usepackage[export]{adjustbox}
\usepackage[backend=bibtex,style=numeric]{biblatex}

\bibliography{String_Theory.bib}

\usetheme{Madrid}

\usefonttheme{professionalfonts}

\setbeamertemplate{itemize item}[triangle]
\setbeamertemplate{itemize subsubitem}[square]
\setbeamertemplate{enumerate item}[default]
\setbeamertemplate{enumerate subitem}[default]
\setbeamertemplate{bibliography item}{\insertbiblabel}

\begin{document}	
	\author{Mate Zoltan Farkas}
	\title{An Introduction to String Theory}
	%\subtitle{}
	%\logo{}
	%\institute{}
	%\date{}
	%\subject{}
	\setbeamercovered{transparent}
	\setbeamertemplate{navigation symbols}{}
	
	\begin{frame}[plain]
		\maketitle
	\end{frame}
	
	\begin{frame}
		\frametitle{Table of Contents}
		\tableofcontents
	\end{frame}

	\section{Introduction}
	\subsection{Notation}
	
	\begin{frame}
		\frametitle{Introduction -- Notation}
		\begin{itemize}
			\item Metric:
			\begin{itemize}
				\item $\eta_{\mu\nu}$ = diag(-1,+1, $\dots$, +1)
			\end{itemize}
		\end{itemize}
	\end{frame}

	\section{The Relativistic String}
	\subsection{The Relativistic Point Particle}
	
	\begin{frame}
		\frametitle{The Relativistic Point Particle -- The Action}
		\begin{itemize}[]
			\item<1-> Consider the action of a point particle (with fixed coordinates $X_\mu = (t,\vec{x})$ in a given frame):
			\begin{equation*}
				S = -m \int dt \, \sqrt{1-\dot{\vec{x}}\dot{\vec{x}}}
			\end{equation*}
			\item<1->[] $\rightarrow$ not Lorentz-invariant, due to mixture of spacial and temporal coordinates under a Lorentz-transformation $\Lambda$.
			\item[]
			\item<2-> Consider instead for a generalized coordinate $\tau$ along the line element:
			\begin{columns}
				\begin{column}{0.8\textwidth}
					\begin{equation*}
						S = -m \int d\tau \, \sqrt{-\frac{dX^\mu}{d\tau} \frac{dX^\nu}{d\tau}\eta_{\mu\nu}}
					\end{equation*}
				\end{column}
				\begin{column}{0.2\textwidth}
					\begin{figure}
						\includegraphics[width=0.5\linewidth]{res/TongP17_graph}
						\caption*{Ref: \cite{tong_lectures_2012}}
					\end{figure}
				\end{column}
			\end{columns}
			\item[]<2-> Remark: $S$ is proportional to the integral over the worldline of the particle
		\end{itemize}
	\end{frame}

	\begin{frame}
		\frametitle{The Relativistic Point Particle -- Action Symmetries}
		\begin{itemize}
			\item[]<1-> What are the symmetries of
			\begin{equation*}
				S = -m \int d\tau \, \sqrt{-\frac{dX^\mu}{d\tau} \frac{dX^\nu}{d\tau}\eta_{\mu\nu}}
			\end{equation*}
			\item<2-> Reparametrization invariance:
			\begin{flushleft}
				Let $\tilde{\tau} = \tilde{\tau}(\tau)$. Then:
				\begin{equation*}
					S' = -m \int d\tilde{\tau} \, \sqrt{-\frac{dX^\mu}{d\tilde{\tau}}\frac{dX^\nu}{d\tilde{\tau}}\eta_{\mu\nu}} = S
				\end{equation*}
				$\rightarrow$ gauge symmetry of the action $\rightarrow$ still D-1 dof!
			\end{flushleft}
			\item<2-> Poincaré invariance:
			\begin{flushleft}
				Let:
				\begin{equation*}
					X'^\mu = \Lambda^\mu{}_\nu X^\nu + c^\mu
				\end{equation*}
				Then:
				\begin{equation*}
					S'=S, \qquad \text{as} \qquad \Lambda^\mu{}_\rho \, \eta_{\mu\nu} \, \Lambda^\nu{}_\sigma = \eta_{\rho\sigma}
				\end{equation*}
			\end{flushleft}
		\end{itemize}
	\end{frame}

	\subsection{Dynamics of a Relativistic String}
	\subsubsection{The Nambu-Goto Action}

	\begin{frame}[t]
		\frametitle{Dynamics of a Relativistic String -- The Nambu-Goto Action}
		Action of the Relativistic String?
		\begin{itemize}
			\item[]<only@1>
			\begin{center}
				\begin{tabular}{ccp{1.8cm}cc}
					particle & $\Leftrightarrow$ & worldline & $\Leftrightarrow$ & ${\displaystyle S = -m \int } \underbrace{d\tau \,\sqrt{-\frac{dX^\mu}{d\tau} \frac{dX^\nu}{d\tau}\eta_{\mu\nu}} }_{\text{line element }ds} $ \\[0.7cm]
					&&&&\\
					closed string & $\Leftrightarrow$ &\hphantom{worl} ? \hphantom{heet}& $\Leftrightarrow$ & ? \\
				\end{tabular}
			\end{center}
			\item[]<only@2>
			\begin{center}
				\begin{tabular}{ccp{1.8cm}cc}
					particle & $\Leftrightarrow$ & worldline & $\Leftrightarrow$ & ${\displaystyle S = -m \int d\tau \,\sqrt{-\frac{dX^\mu}{d\tau} \frac{dX^\nu}{d\tau}\eta_{\mu\nu}} }$ \\[0.7cm]
					&&&&\\
					closed string & $\Leftrightarrow$ & worldsheet & $\Leftrightarrow$ & ? \\
				\end{tabular}
			\end{center}
			\item[]<only@3>
			\begin{center}
				\begin{tabular}{ccp{1.8cm}cc}
					particle & $\Leftrightarrow$ & worldline & $\Leftrightarrow$ & ${\displaystyle S = -m \int d\tau \,\sqrt{-\frac{dX^\mu}{d\tau} \frac{dX^\nu}{d\tau}\eta_{\mu\nu}} }$ \\[0.7cm]
					&&&&\\
					closed string & $\Leftrightarrow$ & worldsheet & $\Leftrightarrow$ & Nambu-Goto/Dirac Action \\
				\end{tabular}
			\end{center}
			\item[]<3->
			Boundary condition of closed strings living in $D$ dimensions:
			\begin{equation*}
				X^\mu(\tau,\sigma) = X^\mu(\tau,\sigma+2\pi) \quad  \text{for $\mu$} = 0,1,...,D-1; \, \sigma \in \left[0;2\pi\right)
			\end{equation*}
			Shorthand notation: $\sigma^\alpha = (\tau,\sigma)$ for $\alpha \in \{0;1\}$\\
			\color{red} \textbf{image of a string}
		\end{itemize}
	\end{frame}
	
	\begin{frame}
		\frametitle{Dynamics of a Relativistic String -- The Nambu-Goto Action}
		How to describe the surface area to construct the action?
		\begin{equation*}
			S \propto \int\displaylimits_{\partial V} d^2x = \int\displaylimits_{\partial V}d^2\sigma \left| \det J \right|
		\end{equation*}
		\color{red}{For this, first remember that any metric is given by:}
		\begin{equation*}
			g_{\alpha\beta} = g(e_\alpha,e_\beta)
		\end{equation*}
		\color{black}
		So the metric on the worldsheet (the pullback metric) can be written as:
		\begin{equation*}
			\gamma_{\alpha\beta} = \underbrace{\frac{\partial X^\mu}{\partial\sigma^\alpha} \frac{\partial X^\nu}{\partial\sigma^\beta} \eta_{\mu\nu}}_{(J^T \eta J)_{\alpha\beta}}
		\end{equation*}
		Thus,
		\begin{equation*}
			\det \gamma = \det\eta \, \det{^2} J = - \det{^2} J
		\end{equation*}
		\begin{equation*}
			\left| \det J \right| = \sqrt{-\det\gamma} = \sqrt{-\gamma}
		\end{equation*}
	\end{frame}

	\begin{frame}
		\frametitle{Dynamics of a Relativistic String -- The Nambu-Goto Action}
		So write the action as:
		\begin{equation*}
			S_{NG} = -\frac{1}{2\pi\alpha'} \int d^2\sigma \, \sqrt{-\gamma}
		\end{equation*}
		This action in invariant under,
		\begin{itemize}
			\item Poincaré transformations
			\item Reparametrization
		\end{itemize}
		and the eqations of motion (EoM) are given by:
		\begin{equation*}
			\partial_\alpha\left(\sqrt{-\gamma}\,\gamma^{\alpha\beta}\partial_\beta X^\mu\right) = 0
		\end{equation*}
		$\Rightarrow$ rather hard to quantize in this form! Alternatives?
	\end{frame}

	\subsubsection{The Polyakov Action}

	\begin{frame}
		\frametitle{Dynamics of a Relativistic String -- The Polyakov Action}
		Way out: \textbf{The Polyakov Action}:
		\begin{equation*}
			S_P = -\frac{1}{4\pi\alpha'}\int d^2 \, \sigma \sqrt{-g} \, g^{\alpha\beta}\,\partial_\alpha X^\mu \partial_\beta X^\nu \, \eta_{\mu\nu}
		\end{equation*}
		The EoMs:
		\begin{itemize}
			\item for $X^\mu$:
			\begin{itemize}
				\item same as for the Nambu-Goto action!
			\end{itemize}
			\begin{equation*}
				\partial_\alpha\left(\sqrt{-g}\,g^{\alpha\beta}\partial_\beta X^{\mu}\right) = 0
			\end{equation*}
			\item for $g_{\alpha\beta}$:
			\begin{equation*}
				g_{\alpha\beta} = 2 \frac{\partial_\alpha X^\mu \partial_\beta X^\nu \eta_{\mu\nu} }{g^{\rho\sigma}\partial_\rho X^\mu \partial_\sigma X^\nu \eta_{\mu\nu}} \equiv 2 \frac{\partial_\alpha X \partial_\beta X}{g^{\rho\sigma}\partial_\rho X \partial_\sigma X}
			\end{equation*}
		\end{itemize}
		Gains of the Polyakov action?
	\end{frame}

	\begin{frame}
		\frametitle{Dynamics of a Relativistic String -- The Polyakov Action}
		Symmetries of the Polyakov action
		\begin{equation*}
			S_P = -\frac{1}{4\pi\alpha'}\int d^2 \, \sigma \sqrt{-g} \, g^{\alpha\beta}\,\partial_\alpha X^\mu \partial_\beta X^\nu \, \eta_{\mu\nu}
		\end{equation*}
		\begin{itemize}
			\item Poincaré invariance
			\item Reparametrization invariance
			\item Invariance under:
			\begin{equation*}
				g'_{\alpha\beta} = \Omega^2(\sigma) g_{\alpha\beta}
			\end{equation*}
			$\Rightarrow$ \textbf{Weyl invariance}\\
			However, using the conformal gauge and writing $g$ as
			\begin{equation*}
				g_{\alpha\beta} = e^{2\phi}\eta_{\alpha\beta}
			\end{equation*}
			can be undone by a Weyl transformation ($\phi=0$). Thus,
			\begin{equation*}
				g_{\alpha\beta} = \eta_{\alpha\beta}
			\end{equation*}
			\color{red} Plot of the transformation in action
		\end{itemize}
	\end{frame}

	\section{Equation of Motion of the Polyakov Action}

	\begin{frame}
		\frametitle{Equation of Motion of the Polyakov Action}
		With $g_{\alpha\beta} = \eta_{\alpha\beta}$ the EoMs
		\begin{equation*}
			\begin{cases}
				\partial_\alpha\left(\sqrt{-g}\,g^{\alpha\beta}\partial_\beta X^{\mu}\right) = 0 \\
				g_{\alpha\beta} =  2 \frac{\partial_\alpha X \partial_\beta X}{g^{\rho\sigma}\partial_\rho X \partial_\sigma X}
			\end{cases}
		\end{equation*}
		are reduced to:
		\begin{equation*}
			\begin{cases}
				\partial_\alpha\partial^\alpha X^{\mu} & = 0 \\
				T_{\alpha\beta} &= \partial_\alpha X \partial_\beta X - \frac{1}{2} \eta_{\alpha\beta} \eta^{\mu\nu} \partial_\mu X \partial_\nu X = 0
			\end{cases}
		\end{equation*}
		With the second constraints explicitly as:
		\begin{equation*}
			\begin{cases}
			T_{01} &= \dot{X}\cdot X' = 0 \\
			T_{00} &= T_{11} = \frac{1}{2} \left(\dot{X}^2 + X'^2 \right) = 0
			\end{cases}
		\end{equation*}
	\end{frame}	

	\begin{frame}
		\frametitle{Solution to the EoMs}
		To find a solution, define the lightcone coordinates as
		\begin{equation*}
			\sigma^\pm = \tau\pm\sigma
		\end{equation*}
		Then, the EoMs are given by
		\begin{equation*}
			\begin{cases}
				\partial_+\partial_-X^\mu &= 0 \\
				(\partial_+X)^2 &= 0 \qquad, \text{with} \, X^\mu(\tau,\sigma) = X^\mu(\tau,\sigma+2\pi) \\
				(\partial_-X)^2 &= 0 \\
			\end{cases}
		\end{equation*}
		\begin{align*}
			\Rightarrow X^\mu(\tau,\sigma) & = X^\mu_L(\sigma^+) + X^\mu_R(\sigma^-) \\
			X^\mu_{L}(\sigma^+) &= \frac{1}{2}x^\mu + \frac{1}{2} \alpha' p^\mu \sigma^+ + i\sqrt{\frac{\alpha'}{2}} \sum_{n\neq 0} \frac{1}{n} \tilde{\alpha}^\mu e^{-in\sigma^+} \\
			X^\mu_{R}(\sigma^-) &= \frac{1}{2}x^\mu + \frac{1}{2} \alpha' p^\mu \sigma^- + i\sqrt{\frac{\alpha'}{2}} \sum_{n\neq 0} \frac{1}{n} \alpha^\mu e^{-in\sigma^-} \\
		\end{align*}
	\end{frame}

	\begin{frame}
		\frametitle{Solution to the EoMs}
		The constraints $\left(\partial_\pm X\right)^2 = 0$:
		\begin{itemize}
			\item First:
			\begin{align*}
				\partial_- X^\mu = \frac{\alpha'}{2} p^\mu + \sqrt{\frac{\alpha'}{2}} \sum_{n\neq 0 } \alpha^\mu_n e^{-in\sigma^-} & \equiv \sum_{n} \alpha^\mu_n e^{-in\sigma^-} \\
				\text{with} \quad \alpha^\mu_0 & = \sqrt{\frac{\alpha'}{2}}p^\mu \\
			\end{align*}
			\item Then:
			\begin{align*}
				\left(\partial_-X\right)^2 &= \frac{\alpha'}{2} \sum_{n,p} \alpha_m \alpha_n e^{-i(n+m)\sigma^-}\\
				& = \alpha' \sum_n \underbrace{\frac{1}{2} \sum_m \alpha_m \alpha_{n-m} e^{-in\sigma^-}}_{L_n} \stackrel{!}{=} 0
			\end{align*}
		\end{itemize}
	\end{frame}

	\begin{frame}
		\frametitle{Solution to the EoMs}
		$\Rightarrow \infty$ number of constraints on $L_n$: 
		\begin{equation*}
			L_n = \tilde{L}_n = 0
		\end{equation*}
		However, $L_0$ contains the mass $M = -p^2$ as:
		\begin{align*}
			L_0 &= \frac{\alpha'}{2}p^2 + \frac{1}{2} \sum_{m>0} \alpha_m \alpha_{-m} e^{-in\sigma^-} \\
			& \Rightarrow M^2 = \frac{4}{\alpha'} \sum_{m>0}\alpha_m\alpha_{-m} = \frac{4}{\alpha'} \sum_{m>0}\tilde{\alpha}_m\tilde{\alpha}_{-m}
		\end{align*}
		$\Rightarrow$ \textbf{Level matching}
	\end{frame}

	\begin{frame}
		\frametitle{Polyakov Action -- Summary}
		{\Huge \color{red} A brief summary before moving to the covariant quantization}
	\end{frame}

	\section{Covariant Quantization of the  Classical Solutions}	

	\section{Quantization in the Lightcone Gauge}
	
	\begin{frame}
		\frametitle{Quantization of $X^\mu$ in the Lightcone Gauge}
	\end{frame}

	\begin{frame}
		\frametitle{References}
		\printbibliography
	\end{frame}

\end{document}